\documentclass{article}
\begin{document}

% ==================================================================
% PREGUNTA 1
% (Tema: lógica, proposiciones, dif:1, res:a, week:3)
% ==================================================================
\begin{question}{1}{lógica, proposiciones}{1}{a}{3}{
\textbf{Suponga que:} \medskip

\(p = \) ``Llovió ayer por la noche".\\
\(q = \) ``Se prendieron los rociadores de pasto ayer por la noche".\\
\(r = \) ``El pasto estaba mojado hoy por la mañana".\medskip

\textbf{Traduzca al español la siguiente proposición:}
\[
\neg p
\]

\begin{enumerate}
    \item a) No llovió ayer por la noche.
    \item b) Se prendieron los rociadores de pasto ayer por la noche.
    \item c) Llovió ayer por la noche.
    \item d) El pasto estaba mojado hoy por la mañana.
\end{enumerate}
}
\end{question}

% ==================================================================
% PREGUNTA 2
% (Tema: lógica, proposiciones, dif:1, res:b, week:3)
% ==================================================================
\begin{question}{2}{lógica, proposiciones}{1}{b}{3}{
\textbf{Suponga que:} \medskip

\(p = \) ``Llovió ayer por la noche".\\
\(q = \) ``Se prendieron los rociadores de pasto ayer por la noche".\\
\(r = \) ``El pasto estaba mojado hoy por la mañana".\medskip

\textbf{Traduzca al español la siguiente proposición:}
\[
r \land \neg p
\]

\begin{enumerate}
    \item a) El pasto estaba mojado hoy por la mañana o no llovió ayer por la noche.
    \item b) El pasto estaba mojado hoy por la mañana y no llovió ayer por la noche.
    \item c) No se prendieron los rociadores de pasto ayer por la noche y llovió ayer por la noche.
    \item d) El pasto no estaba mojado hoy por la mañana, pero llovió ayer por la noche.
\end{enumerate}
}
\end{question}

% ==================================================================
% PREGUNTA 3
% (Tema: lógica, proposiciones, dif:1, res:a, week:3)
% ==================================================================
\begin{question}{3}{lógica, proposiciones}{1}{a}{3}{
\textbf{Suponga que:} \medskip

\(p = \) ``Llovió ayer por la noche".\\
\(q = \) ``Se prendieron los rociadores de pasto ayer por la noche".\\
\(r = \) ``El pasto estaba mojado hoy por la mañana".\medskip

\textbf{Traduzca al español la siguiente proposición:}
\[
(p \land q) \lor \neg p
\]

\begin{enumerate}
   \item a) Llovió ayer por la noche y se prendieron los rociadores de pasto ayer por la noche, o no llovió ayer por la noche.
   \item b) El pasto estaba mojado hoy por la mañana y llovió ayer por la noche, o no se prendieron los rociadores de pasto ayer por la noche.
   \item c) No llovió ayer por la noche y se prendieron los rociadores de pasto ayer por la noche, o el pasto estaba mojado hoy por la mañana.
   \item d) Llovió ayer por la noche o el pasto estaba mojado hoy por la mañana, pero no se prendieron los rociadores de pasto ayer por la noche.
\end{enumerate}
}
\end{question}

% ==================================================================
% PREGUNTA 4
% (Tema: lógica, proposiciones, dif:2, res:c, week:3)
% ==================================================================
\begin{question}{4}{lógica, proposiciones}{2}{c}{3}{
\textbf{Suponga que:} \medskip

\(p = \) ``Llovió ayer por la noche".\\
\(q = \) ``Se prendieron los rociadores de pasto ayer por la noche".\\
\(r = \) ``El pasto estaba mojado hoy por la mañana".\medskip

\textbf{Traduzca al español la siguiente proposición:}
\[
(\neg p \land q) \lor (p \land \neg q)
\]

\begin{enumerate}
   \item a) El pasto estaba mojado hoy por la mañana y llovió ayer por la noche, o se prendieron los rociadores de pasto ayer por la noche.
   \item b) Llovió ayer por la noche y el pasto estaba mojado hoy por la mañana, o no se prendieron los rociadores de pasto ayer por la noche.
   \item c) No llovió ayer por la noche y se prendieron los rociadores de pasto ayer por la noche, o llovió ayer por la noche y no se prendieron los rociadores.
   \item d) No llovió ayer por la noche y el pasto estaba mojado hoy por la mañana, o los rociadores no se prendieron.
\end{enumerate}
}
\end{question}

% ==================================================================
% PREGUNTA 5
% (Tema: lógica, proposiciones, dif:2, res:c, week:3)
% ==================================================================
\begin{question}{5}{lógica, proposiciones}{2}{c}{3}{
\textbf{Suponga que:} \medskip

\(p = \) ``Llovió ayer por la noche".\\
\(q = \) ``Se prendieron los rociadores de pasto ayer por la noche".\\
\(r = \) ``El pasto estaba mojado hoy por la mañana".\medskip

\textbf{Traduzca al español la siguiente proposición:}
\[
(p \land \neg q) \lor r
\]

\begin{enumerate}
   \item a) Se prendieron los rociadores de pasto ayer por la noche y llovió ayer por la noche, o el pasto estaba mojado hoy por la mañana.
   \item b) No llovió ayer por la noche y se prendieron los rociadores de pasto ayer por la noche, o el pasto estaba seco hoy por la mañana.
   \item c) Llovió ayer por la noche y no se prendieron los rociadores de pasto ayer por la noche, o el pasto estaba mojado hoy por la mañana.
   \item d) No llovió ayer por la noche y el pasto estaba mojado hoy por la mañana, pero los rociadores no se prendieron.
\end{enumerate}
}
\end{question}

% ==================================================================
% PREGUNTA 6
% (Tema: lógica, proposiciones, dif:2, res:a, week:3)
% ==================================================================
\begin{question}{6}{lógica, proposiciones}{2}{a}{3}{
\textbf{Suponga que:} \medskip

\(p = \) ``Llovió ayer por la noche".\\
\(q = \) ``Se prendieron los rociadores de pasto ayer por la noche".\\
\(r = \) ``El pasto estaba mojado hoy por la mañana".\medskip

\textbf{Traduzca al español la siguiente proposición:}
\[
\neg [(p \land \neg q) \lor \neg p]
\]

\begin{enumerate}
   \item a) Llovió ayer por la noche y se prendieron los rociadores de pasto ayer por la noche.  
   \item b) Llovió ayer por la noche o los rociadores no se prendieron.  
   \item c) No llovió ayer por la noche y el pasto estaba mojado hoy por la mañana.  
   \item d) El pasto estaba mojado hoy por la mañana y los rociadores se prendieron.  
\end{enumerate}
}
\end{question}

% ==================================================================
% PREGUNTA 7
% (Tema: lógica, proposiciones, dif:2, res:a, week:3)
% ==================================================================
\begin{question}{7}{lógica, proposiciones}{2}{a}{3}{
\textbf{Suponga que:} \medskip

\(p = \) ``Llovió ayer por la noche".\\
\(q = \) ``Se prendieron los rociadores de pasto ayer por la noche".\\
\(r = \) ``El pasto estaba mojado hoy por la mañana".\medskip

\textbf{Traduzca al español la siguiente proposición:}
\[
(\neg p \land q) \lor (\neg p \lor q)
\]

\begin{enumerate}
   \item a) No llovió ayer por la noche y se prendieron los rociadores de pasto ayer por la noche, o no llovió ayer por la noche o los rociadores se prendieron.  
   \item b) Llovió ayer por la noche y no se prendieron los rociadores de pasto ayer por la noche, o el pasto estaba mojado hoy por la mañana.  
   \item c) El pasto estaba mojado hoy por la mañana y llovió ayer por la noche, o se prendieron los rociadores de pasto ayer por la noche.  
   \item d) No llovió ayer por la noche y el pasto estaba mojado hoy por la mañana, pero los rociadores no se prendieron.  
\end{enumerate}
}
\end{question}

% ==================================================================
% PREGUNTA 8
% (Tema: lógica, proposiciones, dif:2, res:d, week:3)
% ==================================================================
\begin{question}{8}{lógica, proposiciones}{2}{d}{3}{
\textbf{Suponga que:} \medskip

\(p = \) ``Llovió ayer por la noche".\\
\(q = \) ``Se prendieron los rociadores de pasto ayer por la noche".\\
\(r = \) ``El pasto estaba mojado hoy por la mañana".\medskip

\textbf{Traduzca al español la siguiente proposición:}
\[
\neg [(p \lor q) \land \neg q]
\]

\begin{enumerate}
   \item a) No llovió ayer por la noche y el pasto estaba mojado hoy por la mañana. 
   \item b) Llovió ayer por la noche y se prendieron los rociadores de pasto ayer por la noche.  
   \item c) El pasto estaba mojado hoy por la mañana y los rociadores no se prendieron.  
   \item d) No llovió ayer por la noche o se prendieron los rociadores de pasto ayer por la noche. 
\end{enumerate}
}
\end{question}

% ==================================================================
% PREGUNTA 9
% (Tema: lógica, proposiciones, dif:2, res:a, week:3)
% ==================================================================
\begin{question}{9}{lógica, proposiciones}{2}{a}{3}{
\textbf{Suponga que:} \medskip

\(p = \) ``Usted tiene gripa".\\
\(q = \) ``Usted no vino al último examen".\\
\(r = \) ``Usted pasa el curso". \medskip

\textbf{Traduzca al español la siguiente proposición:}
\[
(p \land q)\lor(\neg q \land r)
\]

\begin{enumerate}
    \item a) Usted tiene gripa y no vino al último examen, o usted sí vino al último examen y pasa el curso.
    \item b) Usted no tiene gripa y no vino al último examen, o usted sí vino al último examen y pasa el curso.
    \item c) Usted tiene gripa y vino al último examen, pero no pasa el curso.
    \item d) Usted no vino al último examen y pasa el curso, pero no tiene gripa.
\end{enumerate}
}
\end{question}

% ==================================================================
% PREGUNTA 10
% (Tema: lógica, proposiciones, dif:2, res:a, week:3)
% ==================================================================
\begin{question}{10}{lógica, proposiciones}{2}{a}{3}{
\textbf{Suponga que:} \medskip

\(p = \) ``Usted tiene gripa".\\
\(q = \) ``Usted no vino al último examen".\\
\(r = \) ``Usted pasa el curso". \medskip

\textbf{Traduzca al español la siguiente proposición:}
\[
(p \lor q) \rightarrow \neg r
\]

\begin{enumerate}
   \item a) Si usted tiene gripa o no vino al último examen, entonces no pasa el curso.  
   \item b) Usted pasa el curso solo si tiene gripa o no vino al último examen.  
   \item c) Si usted pasa el curso, entonces tiene gripa o no vino al último examen.  
   \item d) Usted no pasa el curso o no tiene gripa ni faltó al último examen.  
\end{enumerate}
}
\end{question}

% ==================================================================
% PREGUNTA 11
% (Tema: lógica, proposiciones, dif:2, res:b, week:3)
% ==================================================================
\begin{question}{11}{lógica, proposiciones}{2}{b}{3}{
\textbf{Suponga que:} \medskip

\(p = \) ``Usted tiene gripa".\\
\(q = \) ``Usted no vino al último examen".\\
\(r = \) ``Usted pasa el curso". \medskip

\textbf{Traduzca al español la siguiente proposición:}
\[
(\neg p \lor r) \land (q \rightarrow p)
\]

\begin{enumerate}
   \item a) Si usted no tiene gripa o pasa el curso, y si no vino al último examen, entonces tiene gripa.  
   \item b) Usted no tiene gripa o pasa el curso, y si usted no vino al último examen, entonces tenía gripa.  
   \item c) Usted no tiene gripa o pasa el curso, y si no vino al último examen, entonces usted no pasa el curso.  
   \item d) Si usted no tiene gripa y pasa el curso, entonces usted no vino al último examen.  
\end{enumerate}
}
\end{question}

% ==================================================================
% PREGUNTA 12
% (Tema: lógica, proposiciones, dif:2, res:c, week:3)
% ==================================================================
\begin{question}{12}{lógica, proposiciones}{2}{c}{3}{
\textbf{Suponga que:} \medskip

\(p = \) ``Usted tiene gripa".\\
\(q = \) ``Usted no vino al último examen".\\
\(r = \) ``Usted pasa el curso". \medskip

\textbf{Traduzca al español la siguiente proposición:}
\[
\neg [(p \lor q) \land \neg r]
\]

\begin{enumerate}
   \item a) Si usted tiene gripa o no vino al último examen, entonces pasa el curso.  
   \item b) No es cierto que usted tenga gripa o no vino al último examen, y no pasa el curso.  
   \item c) Usted pasa el curso o no tiene gripa y vino al último examen.  
   \item d) Usted no tiene gripa ni faltó al último examen, pero pasa el curso.  
\end{enumerate}
}
\end{question}

% ==================================================================
% PREGUNTA 13
% (Tema: lógica, proposiciones, dif:2, res:a, week:3)
% ==================================================================
\begin{question}{13}{lógica, proposiciones}{2}{a}{3}{
\textbf{Suponga que:} \medskip

\(p = \) ``Usted tiene gripa".\\
\(q = \) ``Usted no vino al último examen".\\
\(r = \) ``Usted pasa el curso". \medskip

\textbf{Traduzca al español la siguiente proposición:}
\[
(p \lor q) \leftrightarrow \neg r
\]

\begin{enumerate}
   \item a) Usted tiene gripa o no vino al último examen si y solo si no pasa el curso.  
   \item b) Si usted no pasa el curso, entonces tuvo gripa o no vino al último examen.  
   \item c) Si usted tiene gripa o no vino al último examen, entonces no pasa el curso.  
   \item d) Si usted pasa el curso, entonces tuvo gripa o no vino al último examen.  
\end{enumerate}
}
\end{question}

% ==================================================================
% PREGUNTA 14
% (Tema: lógica, proposiciones, dif:3, res:d, week:3)
% ==================================================================
\begin{question}{14}{lógica, proposiciones}{3}{d}{3}{
\textbf{Suponga que:} \medskip

\(p = \) ``Usted tiene gripa".\\
\(q = \) ``Usted no vino al último examen".\\
\(r = \) ``Usted pasa el curso". \medskip

\textbf{Traduzca al español la siguiente proposición:}
\[
[(p \lor \neg q) \rightarrow \neg r] \lor [(r \land p) \rightarrow q]
\]

\begin{enumerate}
   \item a) Si usted pasa el curso o no tuvo gripa, entonces vino al último examen; o si usted pasa el curso y tuvo gripa, entonces faltó al último examen.  
   \item b) Si usted tiene gripa o vino al último examen, entonces pasa el curso; o si usted pasa el curso y tiene gripa, entonces no faltó al último examen.  
   \item c) Si usted no pasa el curso, entonces tuvo gripa o faltó al último examen; o si usted pasa el curso y tuvo gripa, entonces no vino al último examen.  
   \item d) Si usted tiene gripa o no faltó al último examen, entonces no pasa el curso; o si usted pasa el curso y tiene gripa, entonces no faltó al último examen.
\end{enumerate}
}
\end{question}

% ==================================================================
% PREGUNTA 15
% (Tema: conjuntos, operaciones de conjuntos, dif:1, res:d, week:2)
% ==================================================================
\begin{question}{15}{conjuntos, operaciones de conjuntos}{1}{d}{2}{
Para celebrar San Valentín, el CONSEFE organizó CUPICONSEFE que es un domicilio de regalos entre los estudiantes, los administrativos y los profesores de la Facultad. Definimos como el universo a los estudiantes, los administrativos y los profesores de la Facultad de Economía. \textbf{Considere los siguientes conjuntos:}
\[
\begin{aligned}
R &= \{x \in U \mid x \text{ recibió un regalo por CUPICONSEFE}\}\\
E &= \{x \in U \mid x \text{ envió un regalo por CUPICONSEFE}\}\\
Est &= \{x \in U \mid x \text{ es estudiante}\}\\
Prof &= \{x \in U \mid x \text{ es profesor}\}\\
Adm &= \{x \in U \mid x \text{ es administrativo}\}
\end{aligned}
\]

\textbf{Traduzca al español la siguiente proposición:}
\[
R \cap E \neq \emptyset
\]

\begin{enumerate}
    \item a) No hay personas que hayan recibido y enviado regalos al mismo tiempo.
    \item b) Al menos una persona que recibió un regalo no lo envió.
    \item c) Todas las personas enviaron y recibieron regalos.
    \item d) Al menos una persona envió y recibió un regalo.
\end{enumerate}
}
\end{question}

% ==================================================================
% PREGUNTA 16
% (Tema: conjuntos, operaciones de conjuntos, dif:3, res:b, week:2)
% ==================================================================
\begin{question}{16}{conjuntos, operaciones de conjuntos}{3}{b}{2}{
Para celebrar San Valentín, el CONSEFE organizó CUPICONSEFE que es un domicilio de regalos entre los estudiantes, los administrativos y los profesores de la Facultad. Definimos como el universo a los estudiantes, los administrativos y los profesores de la Facultad de Economía. \textbf{Considere los siguientes conjuntos:}
\[
\begin{aligned}
R &= \{x \in U \mid x \text{ recibió un regalo por CUPICONSEFE}\},\\
E &= \{x \in U \mid x \text{ envió un regalo por CUPICONSEFE}\},\\
Est &= \{x \in U \mid x \text{ es estudiante}\},\\
Prof &= \{x \in U \mid x \text{ es profesor}\},\\
Adm &= \{x \in U \mid x \text{ es administrativo}\}.
\end{aligned}
\]

\textbf{Traduzca al español la siguiente proposición:}
\[
\neg (Adm \subseteq [(R\setminus E) \cup (E\setminus R)])
\]

\begin{enumerate}
    \item a) Ningún administrativo recibió o envió un regalo, pero no ambos.
    \item b) Al menos un administrativo envió y recibió un regalo.
    \item c) Todos los administrativos recibieron y enviaron un regalo.
    \item d) Ningún administrativo recibió y envió un regalo al mismo tiempo.
\end{enumerate}
}
\end{question}

% ==================================================================
% PREGUNTA 17
% (Tema: conjuntos, operaciones de conjuntos, dif:2, res:b, week:2)
% ==================================================================
\begin{question}{17}{conjuntos, operaciones de conjuntos}{2}{b}{2}{
Para celebrar San Valentín, el CONSEFE organizó CUPICONSEFE que es un domicilio de regalos entre los estudiantes, los administrativos y los profesores de la Facultad. Definimos como el universo a los estudiantes, los administrativos y los profesores de la Facultad de Economía. \textbf{Considere los siguientes conjuntos:}
\[
\begin{aligned}
R &= \{x \in U \mid x \text{ recibió un regalo por CUPICONSEFE}\},\\
E &= \{x \in U \mid x \text{ envió un regalo por CUPICONSEFE}\},\\
Est &= \{x \in U \mid x \text{ es estudiante}\},\\
Prof &= \{x \in U \mid x \text{ es profesor}\},\\
Adm &= \{x \in U \mid x \text{ es administrativo}\}.
\end{aligned}
\]

\textbf{Use únicamente los conjuntos definidos para escribir una expresión equivalente a:} \medskip

\textit{“Sólo los profesores que también son estudiantes recibieron regalos.”} \medskip

\begin{enumerate}
    \item a) \( R = Prof \cap Est \)  
    \item b) \( R \subseteq (Prof \cap Est) \)  
    \item c) \( R = (Prof \cap Est) \cup Adm \)  
    \item d) \( R \subseteq (Prof \cup Est) \)  
\end{enumerate}
}
\end{question}

% ==================================================================
% PREGUNTA 18
% (Tema: conjuntos, operaciones de conjuntos, dif:2, res:a, week:2)
% ==================================================================
\begin{question}{18}{conjuntos, operaciones de conjuntos}{2}{a}{2}{
\textbf{Considere el conjunto de películas que se presentan en los cines de Bogotá como universo \(U\).}\\
\textbf{Sean:}
\[
\begin{aligned}
CC &= \{x \in U \mid x\text{ se presenta en Cine Colombia}\},\\
CM &= \{x \in U \mid x\text{ se presenta en Cine-Mark}\}.
\end{aligned}
\]
\textbf{Use únicamente los conjuntos definidos para escribir una expresión equivalente a:} \medskip

\textit{“Todas las películas que se presentan en Bogotá, están en Cine Colombia pero no en Cine-Mark, o están en Cine-Mark pero no en Cine Colombia.”} \medskip

\begin{enumerate}
    \item a) \( U = (CC \setminus CM) \cup (CM \setminus CC) \)  
    \item b) \( U = CC \cup CM \)  
    \item c) \( U \subseteq (CC \cap CM) \)  
    \item d) \( U \subseteq (CC \cup CM) \)  
\end{enumerate}
}
\end{question}

% ==================================================================
% PREGUNTA 19
% (Tema: conjuntos, operaciones de conjuntos, dif:2, res:c, week:2)
% ==================================================================
\begin{question}{19}{conjuntos, operaciones de conjuntos}{2}{c}{2}{
\textbf{Considere el conjunto de películas que se presentan en los cines de Bogotá como universo \(U\).}\\
\textbf{Sean:}
\[
\begin{aligned}
CC &= \{x \in U \mid x\text{ se presenta en Cine Colombia}\},\\
CM &= \{x \in U \mid x\text{ se presenta en Cine-Mark}\}.
\end{aligned}
\]
\textbf{Traduzca al español la siguiente proposición:}
\[
(CC \cup CM)^c = CC^c \cap CM^c
\]

\begin{enumerate}
    \item a) No hay películas en Bogotá fuera de Cine Colombia y Cine-Mark.  
    \item b) Todas las películas en Bogotá están en al menos uno de los dos cines.  
    \item c) Las películas que no están en Cine Colombia ni en Cine-Mark son exactamente las que están fuera del universo.  
    \item d) Todas las películas de Bogotá que no están en Cine Colombia tampoco están en Cine-Mark.  
\end{enumerate}
}
\end{question}

% ==================================================================
% PREGUNTA 20
% (Tema: conjuntos, operaciones de conjuntos, dif:2, res:c, week:2)
% ==================================================================
\begin{question}{20}{conjuntos, operaciones de conjuntos}{2}{c}{2}{
\textbf{Sea U el conjunto de los equipos de la NFL. Considere los siguientes conjuntos:}
\[
\begin{aligned}
AFC &= \{ x \in U \mid x \text{ hace parte de la American Football Conference}\},\\
NFC &= \{ x \in U \mid x \text{ hace parte de la National Football Conference}\},\\
Div &= \{ x \in U \mid x \text{ ganó su división}\},\\
Top_6 &= \{ x \in U \mid x \text{ está en el top 6 de su conferencia}\},\\
Top_2 &= \{ x \in U \mid x \text{ está en el top 2 de su conferencia}\},\\
PlayOFFs &= \{ x \in U \mid x \text{ pasó a los play-offs}\},\\
SB &= \{ x \in U \mid x \text{ juega en el Super Bowl}\}.
\end{aligned}
\]

\textbf{Use únicamente los conjuntos definidos para escribir una expresión equivalente a:} 
\[
\bigl[(x \in Top_6)\ \land\ (x \not\in Div)\bigr]\ \lor\ \bigl(x \in Div\bigr)
\]

\begin{enumerate}
    \item a) \((Top_6 \cap Div) \cup Div\)
    \item b) \((Top_6 \cup Div) \cap Div\)
    \item c) \((Top_6 \setminus Div)\ \cup\ Div\)
    \item d) \((Top_6 \setminus Div)\ \cap\ Div\)
\end{enumerate}
}
\end{question}

% ==================================================================
% PREGUNTA 21
% (Tema: conjuntos, funciones, dif:2, res:a, week:9)
% ==================================================================
\begin{question}{21}{conjuntos, funciones}{2}{a}{9}{
\textbf{Sea U el conjunto de los equipos de la NFL. Considere los siguientes conjuntos:}
\[
\begin{aligned}
AFC &= \{ x \in U \mid x \text{ hace parte de la American Football Conference}\},\\
NFC &= \{ x \in U \mid x \text{ hace parte de la National Football Conference}\},\\
Div &= \{ x \in U \mid x \text{ ganó su división}\},\\
Top_6 &= \{ x \in U \mid x \text{ está en el top 6 de su conferencia}\},\\
Top_2 &= \{ x \in U \mid x \text{ está en el top 2 de su conferencia}\},\\
PlayOFFs &= \{ x \in U \mid x \text{ pasó a los play-offs}\},\\
SB &= \{ x \in U \mid x \text{ juega en el Super Bowl}\}.
\end{aligned}
\]

Para $x \in U$, $d(x)$ denota el turno que le toca al equipo $x$ en el Draft del año siguiente. \medskip

\textbf{Traduzca al español la siguiente proposición:}
\[
(x \in SB \land x \in AFC)\to d(x)>10
\]

\begin{enumerate}
    \item a) Si un equipo está en el Super Bowl y en la AFC, entonces su turno en el Draft es mayor a 10.  
    \item b) Todos los equipos del Super Bowl tienen un turno en el Draft mayor a 10.  
    \item c) Un equipo de la AFC solo puede estar en el Super Bowl si su turno en el Draft es mayor a 10.  
    \item d) Si un equipo tiene un turno en el Draft mayor a 10, entonces juega en el Super Bowl y es de la AFC.  
\end{enumerate}
}
\end{question}

% ==================================================================
% PREGUNTA 22
% (Tema: conjuntos, índices, dif:2, res:a, week:7)
% ==================================================================
\begin{question}{22}{conjuntos, índices}{2}{a}{7}{
\textbf{Sea U el conjunto de los equipos de la NFL. Considere los siguientes conjuntos:}
\[
\begin{aligned}
AFC &= \{ x \in U \mid x \text{ hace parte de la American Football Conference}\},\\
NFC &= \{ x \in U \mid x \text{ hace parte de la National Football Conference}\},\\
Div &= \{ x \in U \mid x \text{ ganó su división}\},\\
Top_6 &= \{ x \in U \mid x \text{ está en el top 6 de su conferencia}\},\\
Top_2 &= \{ x \in U \mid x \text{ está en el top 2 de su conferencia}\},\\
PlayOFFs &= \{ x \in U \mid x \text{ pasó a los play-offs}\},\\
SB &= \{ x \in U \mid x \text{ juega en el Super Bowl}\}.
\end{aligned}
\]

Para $x \in U$, $d(x)$ denota el turno que le toca al equipo $x$ en el Draft del año siguiente. \medskip

\textbf{Traduzca al español la siguiente proposición:}
\[
(x \in Top_6^c)\land(d(x) \in \{1,2,\dots,20\})
\]

\begin{enumerate}
    \item a) Un equipo que no está en el top 6 de su conferencia tiene un turno en el Draft entre 1 y 20.  
    \item b) Un equipo que está en el top 6 de su conferencia tiene un turno en el Draft entre 1 y 20.  
    \item c) Un equipo que no está en el top 6 de su conferencia tiene un turno en el Draft mayor a 20.  
    \item d) Un equipo con turno en el Draft entre 1 y 20 no está en el top 6 de su conferencia.  
\end{enumerate}
}
\end{question}

% ==================================================================
% PREGUNTA 23
% (Tema: conjuntos, índices, dif:2, res:a, week:7)
% ==================================================================
\begin{question}{23}{conjuntos, índices}{2}{a}{7}{
\textbf{Sea U el conjunto de los equipos de la NFL. Considere los siguientes conjuntos:}
\[
\begin{aligned}
AFC &= \{ x \in U \mid x \text{ hace parte de la American Football Conference}\},\\
NFC &= \{ x \in U \mid x \text{ hace parte de la National Football Conference}\},\\
Div &= \{ x \in U \mid x \text{ ganó su división}\},\\
Top_6 &= \{ x \in U \mid x \text{ está en el top 6 de su conferencia}\},\\
Top_2 &= \{ x \in U \mid x \text{ está en el top 2 de su conferencia}\},\\
PlayOFFs &= \{ x \in U \mid x \text{ pasó a los play-offs}\},\\
SB &= \{ x \in U \mid x \text{ juega en el Super Bowl}\}.
\end{aligned}
\]

Para $x \in U$, $d(x)$ denota el turno que le toca al equipo $x$ en el Draft del año siguiente. \medskip

\textbf{Traduzca al español la siguiente proposición:}
\[
(x \in Div \cap Top_2)\land(d(x) \geq 25)
\]

\begin{enumerate}
    \item a) Un equipo que ganó su división y está en el top 2 de su conferencia tiene un turno en el Draft de al menos 25.  
    \item b) Un equipo que ganó su división y está en el top 2 de su conferencia tiene un turno en el Draft menor a 25.  
    \item c) Si un equipo tiene un turno en el Draft mayor o igual a 25, entonces ganó su división y está en el top 2 de su conferencia.  
    \item d) Todos los equipos en el top 2 de su conferencia tienen un turno en el Draft mayor o igual a 25.  
\end{enumerate}
}
\end{question}

% ==================================================================
% PREGUNTA 24
% (Tema: conjuntos, índices, dif:2, res:a, week:7)
% ==================================================================
\begin{question}{24}{conjuntos, índices}{2}{a}{7}{
\textbf{Sea U el conjunto de los equipos de la NFL. Considere los siguientes conjuntos:}
\[
\begin{aligned}
AFC &= \{ x \in U \mid x \text{ hace parte de la American Football Conference}\},\\
NFC &= \{ x \in U \mid x \text{ hace parte de la National Football Conference}\},\\
Div &= \{ x \in U \mid x \text{ ganó su división}\},\\
Top_6 &= \{ x \in U \mid x \text{ está en el top 6 de su conferencia}\},\\
Top_2 &= \{ x \in U \mid x \text{ está en el top 2 de su conferencia}\},\\
PlayOFFs &= \{ x \in U \mid x \text{ pasó a los play-offs}\},\\
SB &= \{ x \in U \mid x \text{ juega en el Super Bowl}\}.
\end{aligned}
\]

Para $x \in U$, $d(x)$ denota el turno que le toca al equipo $x$ en el Draft del año siguiente. \medskip

\textbf{Traduzca al español la siguiente proposición:}
\[
(x \in Div \cap Top_2^c)\land(d(x) \geq 21)
\]

\begin{enumerate}
    \item a) Un equipo que ganó su división pero no está en el top 2 de su conferencia tiene un turno en el Draft de al menos 21.  
    \item b) Un equipo que no ganó su división pero está en el top 2 de su conferencia tiene un turno en el Draft de al menos 21.  
    \item c) Todos los equipos en el top 2 de su conferencia tienen un turno en el Draft mayor o igual a 21.  
    \item d) Un equipo que no ganó su división y no está en el top 2 de su conferencia tiene un turno en el Draft menor a 21.  
\end{enumerate}
}
\end{question}

% ==================================================================
% PREGUNTA 25
% (Tema: cuantificadores, proposiciones, dif:1, res:a, week:6)
% ==================================================================
\begin{question}{25}{cuantificadores, proposiciones}{1}{a}{6}{
Suponga que $X$ es el conjunto de sitios para estudiar en la Universidad de los Andes y sea $P(x)=``x\text{ está lleno}"$.\\

\textbf{Traduzca al español la siguiente proposición:}
\[
\exists x \in X, \text{ tal que } P(x).
\]

\begin{enumerate}
   \item a) Algún sitio para estudiar en los Andes está lleno.  
   \item b) Todos los sitios para estudiar en los Andes están llenos.  
   \item c) Ningún sitio para estudiar en los Andes está lleno.  
   \item d) No hay sitios para estudiar en los Andes.  
\end{enumerate}
}
\end{question}

% ==================================================================
% PREGUNTA 26
% (Tema: cuantificadores, proposiciones, dif:1, res:b, week:6)
% ==================================================================
\begin{question}{26}{cuantificadores, proposiciones}{1}{b}{6}{
Suponga que $X$ es el conjunto de sitios para estudiar en la Universidad de los Andes y sea $P(x)=``x\text{ está lleno}"$.\\

\textbf{Traduzca al español la siguiente proposición:}
\[
\forall x \in X, P(x).
\]

\begin{enumerate}
   \item a) Algún sitio para estudiar en los Andes está lleno.  
   \item b) Todos los sitios para estudiar en los Andes están llenos.  
   \item c) Ningún sitio para estudiar en los Andes está lleno.  
   \item d) Existe al menos un sitio para estudiar en los Andes que no está lleno.  
\end{enumerate}
}
\end{question}

% ==================================================================
% PREGUNTA 27
% (Tema: cuantificadores, proposiciones, dif:1, res:c, week:6)
% ==================================================================
\begin{question}{27}{cuantificadores, proposiciones}{1}{c}{6}{
Suponga que $X$ es el conjunto de sitios para estudiar en la Universidad de los Andes y sea $P(x)=``x\text{ está lleno}"$.\\

\textbf{Traduzca al español la siguiente proposición:}
\[
\exists x \in X, \text{ tal que } \neg P(x).
\]

\begin{enumerate}
   \item a) Todos los sitios para estudiar en los Andes están llenos.  
   \item b) Ningún sitio para estudiar en los Andes está lleno.  
   \item c) Algún sitio para estudiar en los Andes no está lleno.  
   \item d) Todos los sitios para estudiar en los Andes no están llenos.  
\end{enumerate}
}
\end{question}

% ==================================================================
% PREGUNTA 28
% (Tema: cuantificadores, proposiciones, dif:1, res:b, week:6)
% ==================================================================
\begin{question}{28}{cuantificadores, proposiciones}{1}{b}{6}{
Suponga que $X$ es el conjunto de sitios para estudiar en la Universidad de los Andes y sea $P(x)=``x\text{ está lleno}"$.\\

\textbf{Traduzca al español la siguiente proposición:}
\[
\forall x \in X, \neg P(x).
\]

\begin{enumerate}
   \item a) Todos los sitios para estudiar en los Andes están llenos.  
   \item b) Ningún sitio para estudiar en los Andes está lleno.  
   \item c) Algún sitio para estudiar en los Andes está lleno.  
   \item d) No hay sitios para estudiar en los Andes.  
\end{enumerate}
}
\end{question}

% ==================================================================
% PREGUNTA 29
% (Tema: cuantificadores, proposiciones, dif:1, res:b, week:6)
% ==================================================================
\begin{question}{29}{cuantificadores, proposiciones}{1}{b}{6}{
Suponga que $X$ es el conjunto de sitios para estudiar en la Universidad de los Andes y sean $P(x)=``x\text{ está lleno}"$ y $G(x)=``x\text{ tiene conexión a internet}"$.\\

\textbf{Traduzca al español la siguiente proposición:}
\[
\exists x \in X, \text{ tal que } P(x) \land G(x).
\]

\begin{enumerate}
   \item a) Todos los sitios para estudiar en los Andes están llenos y tienen conexión a internet.  
   \item b) Algún sitio para estudiar en los Andes está lleno y tiene conexión a internet.  
   \item c) Ningún sitio para estudiar en los Andes está lleno ni tiene conexión a internet.  
   \item d) Todos los sitios para estudiar en los Andes tienen conexión a internet.  
\end{enumerate}
}
\end{question}

% ==================================================================
% PREGUNTA 30
% (Tema: cuantificadores, proposiciones, dif:2, res:c, week:6)
% ==================================================================
\begin{question}{30}{cuantificadores, proposiciones}{2}{c}{6}{
Suponga que $X$ es el conjunto de sitios para estudiar en la Universidad de los Andes y sean $P(x)=``x\text{ está lleno}"$ y $G(x)=``x\text{ tiene conexión a internet}"$.\\

\textbf{Traduzca al español la siguiente proposición:}
\[
\forall x \in X, P(x) \rightarrow G(x).
\]

\begin{enumerate}
   \item a) Todos los sitios para estudiar en los Andes tienen conexión a internet.  
   \item b) Ningún sitio para estudiar en los Andes tiene conexión a internet.  
   \item c) Si un sitio para estudiar en los Andes está lleno, entonces tiene conexión a internet.  
   \item d) Si un sitio para estudiar en los Andes tiene conexión a internet, entonces está lleno.  
\end{enumerate}
}
\end{question}

% ==================================================================
% PREGUNTA 31
% (Tema: cuantificadores, proposiciones, dif:2, res:a, week:6)
% ==================================================================
\begin{question}{31}{cuantificadores, proposiciones}{2}{a}{6}{
\textbf{Suponga que $X$ es el conjunto de sitios para estudiar en la Universidad de los Andes y sean $P(x)=``x\text{ está lleno}"$ y $G(x)=``x\text{ tiene conexión a internet}"$.}\\

\textbf{Traduzca al español la siguiente proposición:}
\[
\forall x \in X, P(x) \lor \neg G(x).
\]

\begin{enumerate}
    \item a) Todos los sitios para estudiar en los Andes están llenos o no tienen conexión a internet.
    \item b) Ningún sitio para estudiar en los Andes está lleno y todos tienen conexión a internet.
    \item c) Algún sitio para estudiar en los Andes está vacío y tiene conexión a internet.
    \item d) Todos los sitios para estudiar en los Andes tienen conexión a internet y no están llenos.
\end{enumerate}
}
\end{question}

% ==================================================================
% PREGUNTA 32
% (Tema: cuantificadores, proposiciones, dif:2, res:c, week:6)
% ==================================================================
\begin{question}{32}{cuantificadores, proposiciones}{2}{c}{6}{
\textbf{Suponga que $X$ es el conjunto de sitios para estudiar en la Universidad de los Andes y sean $P(x)=``x\text{ está lleno}"$ y $G(x)=``x\text{ tiene conexión a internet}"$.}\\

\textbf{Traduzca al español la siguiente proposición:}
\[
\exists x \in X, \neg P(x) \land G(x).
\]

\begin{enumerate}
    \item a) Todos los sitios para estudiar en los Andes están llenos y no tienen conexión a internet.
    \item b) Ningún sitio para estudiar en los Andes está lleno y todos tienen conexión a internet.
    \item c) Algún sitio para estudiar en los Andes no está lleno y tiene conexión a internet.
    \item d) Todos los sitios para estudiar en los Andes tienen conexión a internet y están llenos.
\end{enumerate}
}
\end{question}

% ==================================================================
% PREGUNTA 33
% (Tema: cuantificadores, proposiciones, dif:2, res:a, week:6)
% ==================================================================
\begin{question}{33}{cuantificadores, proposiciones}{2}{a}{6}{
\textbf{Suponga que $X$ es el conjunto de sitios para estudiar en la Universidad de los Andes y sean $P(x)=``x\text{ está lleno}"$, $G(x)=``x\text{ tiene conexión a internet}"$ y $H(x)=``x\text{ tiene aire acondicionado}"$.}\\

\textbf{Traduzca al español la siguiente proposición:}
\[
(\forall x \in X, (P(x) \land G(x)) \rightarrow (\forall y \in X, H(y) \rightarrow G(y))).
\]

\begin{enumerate}
    \item a) Si un sitio para estudiar en los Andes está lleno y tiene conexión a internet, entonces todos los sitios con aire acondicionado también tienen conexión a internet.
    \item b) Si un sitio para estudiar en los Andes tiene aire acondicionado, entonces tiene conexión a internet.
    \item c) Algún sitio para estudiar en los Andes está lleno y tiene aire acondicionado.
    \item d) Si todos los sitios para estudiar en los Andes tienen conexión a internet, entonces están llenos y tienen aire acondicionado.
\end{enumerate}
}
\end{question}

% ==================================================================
% PREGUNTA 34
% (Tema: cuantificadores, proposiciones, dif:3, res:b, week:6)
% ==================================================================
\begin{question}{34}{cuantificadores, proposiciones}{3}{b}{6}{
\textbf{Suponga que $X$ es el conjunto de sitios para estudiar en la Universidad de los Andes y sean $P(x)=``x\text{ está lleno}"$, $G(x)=``x\text{ tiene conexión a internet}"$ y $H(x)=``x\text{ tiene aire acondicionado}"$.}\\

\textbf{Traduzca al español la siguiente proposición:}
\[
\exists x \in X, (P(x) \land H(x)) \lor (\forall y \in X, \neg G(y) \rightarrow \neg H(y)).
\]

\begin{enumerate}
    \item a) Todos los sitios para estudiar en los Andes tienen aire acondicionado o están llenos.
    \item b) Algún sitio para estudiar en los Andes está lleno y tiene aire acondicionado, o todos los sitios sin conexión a internet no tienen aire acondicionado.
    \item c) Ningún sitio para estudiar en los Andes está lleno o tiene aire acondicionado.
    \item d) Todos los sitios con aire acondicionado están llenos y tienen conexión a internet.
\end{enumerate}
}
\end{question}

% ==================================================================
% PREGUNTA 35
% (Tema: cuantificadores, proposiciones, dif:3, res:a, week:6)
% ==================================================================
\begin{question}{35}{cuantificadores, proposiciones}{3}{a}{6}{
\textbf{Suponga que $X$ es el conjunto de sitios para estudiar en la Universidad de los Andes y sean $P(x)=``x\text{ está lleno}"$, $G(x)=``x\text{ tiene conexión a internet}"$ y $H(x)=``x\text{ tiene aire acondicionado}"$.}\\

\textbf{Traduzca al español la siguiente proposición:}
\[
(\exists x \in X, P(x) \land G(x)) \rightarrow (\exists y \in X, H(y) \land \neg P(y)) \lor (\forall z \in X, G(z)).
\]

\begin{enumerate}
    \item a) Si existe un sitio para estudiar en los Andes que está lleno y tiene conexión a internet, entonces hay un sitio con aire acondicionado que no está lleno o todos los sitios tienen conexión a internet.
    \item b) Si todos los sitios para estudiar en los Andes tienen conexión a internet, entonces todos tienen aire acondicionado.
    \item c) Si hay un sitio para estudiar en los Andes con aire acondicionado, entonces no está lleno ni tiene conexión a internet.
    \item d) Si un sitio para estudiar en los Andes no tiene aire acondicionado, entonces está lleno y tiene conexión a internet.
\end{enumerate}
}
\end{question}

% ==================================================================
% PREGUNTA 36
% (Tema: cuantificadores, proposiciones, dif:1, res:a, week:6)
% ==================================================================
\begin{question}{36}{cuantificadores, proposiciones}{1}{a}{6}{
\textbf{Suponga que $X$ es el conjunto de primíparos de economía y sea $P(x)=``x$ pasa más de 5 horas diarias en clase".}\\

\textbf{Traduzca al español la siguiente proposición:}
\[
\exists x \in X, \text{ tal que } P(x).
\]

\begin{enumerate}
    \item a) Algún primíparo de economía pasa más de 5 horas diarias en clase.
    \item b) Todos los primíparos de economía pasan más de 5 horas diarias en clase.
    \item c) Ningún primíparo de economía pasa más de 5 horas diarias en clase.
    \item d) Todos los primíparos de economía pasan menos de 5 horas diarias en clase.
\end{enumerate}
}
\end{question}

% ==================================================================
% PREGUNTA 37
% (Tema: cuantificadores, proposiciones, dif:1, res:b, week:6)
% ==================================================================
\begin{question}{37}{cuantificadores, proposiciones}{1}{b}{6}{
\textbf{Suponga que $X$ es el conjunto de primíparos de economía y sea $P(x)=``x$ pasa más de 5 horas diarias en clase''.}\\

\textbf{Traduzca al español la siguiente proposición:}
\[
\forall x \in X, P(x).
\]

\begin{enumerate}
\item a) Algún primíparo de economía pasa más de 5 horas diarias en clase.
\item b) Todos los primíparos de economía pasan más de 5 horas diarias en clase.
\item c) Ningún primíparo de economía pasa más de 5 horas diarias en clase.
\item d) Existe al menos un primíparo de economía que pasa menos de 5 horas diarias en clase.
\end{enumerate}
}
\end{question}

% ==================================================================
% PREGUNTA 38
% (Tema: cuantificadores, proposiciones, dif:1, res:c, week:6)
% ==================================================================
\begin{question}{38}{cuantificadores, proposiciones}{1}{c}{6}{
\textbf{Suponga que $X$ es el conjunto de primíparos de economía y sea $P(x)=``x$ pasa más de 5 horas diarias en clase''.}\\

\textbf{Traduzca al español la siguiente proposición:}
\[
\exists x \in X, \text{ tal que } \neg P(x).
\]

\begin{enumerate}
\item a) Todos los primíparos de economía pasan más de 5 horas diarias en clase.
\item b) Ningún primíparo de economía pasa más de 5 horas diarias en clase.
\item c) Algún primíparo de economía pasa menos de 5 horas diarias en clase.
\item d) Todos los primíparos de economía pasan menos de 5 horas diarias en clase.
\end{enumerate}
}
\end{question}

% ==================================================================
% PREGUNTA 39
% (Tema: cuantificadores, proposiciones, dif:1, res:b, week:6)
% ==================================================================
\begin{question}{39}{cuantificadores, proposiciones}{1}{b}{6}{
\textbf{Suponga que $X$ es el conjunto de primíparos de economía y sea $P(x)=``x$ pasa más de 5 horas diarias en clase''.}\\

\textbf{Traduzca al español la siguiente proposición:}
\[
\forall x \in X, \neg P(x).
\]

\begin{enumerate}
\item a) Todos los primíparos de economía pasan más de 5 horas diarias en clase.
\item b) Ningún primíparo de economía pasa más de 5 horas diarias en clase.
\item c) Algún primíparo de economía pasa más de 5 horas diarias en clase.
\item d) No hay primíparos de economía.
\end{enumerate}
}
\end{question}

% ==================================================================
% PREGUNTA 40
% (Tema: cuantificadores, proposiciones, dif:1, res:a, week:6)
% ==================================================================
\begin{question}{40}{cuantificadores, proposiciones}{1}{a}{6}{
\textbf{Suponga que $X$ es el conjunto de primíparos de economía y sean $P(x)=``x$ pasa más de 5 horas diarias en clase'', $G(x)=``x$ estudia al menos 3 horas fuera de clase diariamente'' y $P(x)=``x$ x asiste a todas las clases sin faltar''.}\\

\textbf{Traduzca al español la siguiente proposición:}
\[
\exists x \in X, \text{ tal que } P(x) \land G(x).
\]

\begin{enumerate}
\item a) Algún primíparo de economía pasa más de 5 horas diarias en clase y estudia al menos 3 horas fuera de clase diariamente.
\item b) Todos los primíparos de economía pasan más de 5 horas diarias en clase y estudian al menos 3 horas fuera de clase diariamente.
\item c) Ningún primíparo de economía pasa más de 5 horas diarias en clase ni estudia al menos 3 horas fuera de clase diariamente.
\item d) Todos los primíparos de economía estudian al menos 3 horas fuera de clase diariamente, pero no necesariamente pasan más de 5 horas en clase.
\end{enumerate}
}
\end{question}

% ==================================================================
% PREGUNTA 41
% (Tema: cuantificadores, proposiciones, dif:1, res:a, week:6)
% ==================================================================
\begin{question}{41}{cuantificadores, proposiciones}{1}{a}{6}{
\textbf{Suponga que $X$ es el conjunto de primíparos de economía y sean $P(x)=``x$ pasa más de 5 horas diarias en clase'', $G(x)=``x$ estudia al menos 3 horas fuera de clase diariamente'' y $P(x)=``x$ x asiste a todas las clases sin faltar''.}\\

\textbf{Traduzca al español la siguiente proposición:}
\[
\forall x \in X, P(x) \rightarrow G(x).
\]

\begin{enumerate}
\item a) Algún primíparo de economía que pasa más de 5 horas en clase no estudia fuera de clase.
\item b) Si un primíparo de economía estudia al menos 3 horas fuera de clase, entonces pasa más de 5 horas diarias en clase.
\item c) Todos los primíparos de economía que pasan más de 5 horas en clase también asisten a todas sus clases sin faltar.
\item d) Todos los primíparos de economía que pasan más de 5 horas diarias en clase también estudian al menos 3 horas fuera de clase diariamente.
\end{enumerate}
}
\end{question}

% ==================================================================
% PREGUNTA 42
% (Tema: cuantificadores, proposiciones, dif:2, res:a, week:6)
% ==================================================================
\begin{question}{42}{cuantificadores, proposiciones}{2}{a}{6}{
\textbf{Suponga que $X$ es el conjunto de primíparos de economía y sean $P(x)=``x$ pasa más de 5 horas diarias en clase'', $G(x)=``x$ estudia al menos 3 horas fuera de clase diariamente'' y $P(x)=``x$ x asiste a todas las clases sin faltar''.}\\

\textbf{Traduzca al español la siguiente proposición:}
\[
\forall x \in X, (P(x) \land G(x)) \rightarrow H(x).
\]

\begin{enumerate}
\item a) Si un primíparo de economía pasa más de 5 horas diarias en clase y estudia al menos 3 horas fuera de clase, entonces asiste a todas sus clases sin faltar.
\item b) Si un primíparo de economía pasa más de 5 horas en clase, entonces estudia al menos 3 horas fuera de clase.
\item c) Si un primíparo de economía no estudia al menos 3 horas fuera de clase, entonces no asiste a todas sus clases sin faltar.
\item d) Todos los primíparos de economía que estudian fuera de clase asisten a todas las clases sin faltar.
\end{enumerate}
}
\end{question}

% ==================================================================
% PREGUNTA 43
% (Tema: cuantificadores, proposiciones, dif:2, res:c, week:6)
% ==================================================================
\begin{question}{43}{cuantificadores, proposiciones}{2}{c}{6}{
\textbf{Suponga que $X$ es el conjunto de primíparos de economía y sean $P(x)=``x$ pasa más de 5 horas diarias en clase'', $G(x)=``x$ estudia al menos 3 horas fuera de clase diariamente'' y $P(x)=``x$ x asiste a todas las clases sin faltar''.}\\

\textbf{Traduzca al español la siguiente proposición:}
\[
(\exists x \in X, P(x) \lor H(x)) \rightarrow (\forall y \in X, G(y) \rightarrow P(y)).
\]

\begin{enumerate}
\item a) Si un primíparo de economía pasa más de 5 horas en clase o estudia fuera de clase, entonces todos los primíparos asisten a todas sus clases sin faltar.
\item b) Si todos los primíparos de economía que estudian al menos 3 horas fuera de clase pasan más de 5 horas en clase, entonces todos asisten a todas sus clases sin faltar.
\item c) Si al menos un primíparo de economía pasa más de 5 horas diarias en clase o asiste a todas sus clases sin faltar, entonces todos los primíparos que estudian fuera de clase también pasan más de 5 horas en clase.
\item d) Si algún primíparo de economía no estudia al menos 3 horas fuera de clase, entonces no pasa más de 5 horas en clase.
\end{enumerate}
}
\end{question}

% ==================================================================
% PREGUNTA 44
% (Tema: cuantificadores, proposiciones, dif:3, res:c, week:6)
% ==================================================================
\begin{question}{44}{cuantificadores, proposiciones}{3}{c}{6}{
\textbf{Suponga que $X$ es el conjunto de primíparos de economía y sean $P(x)=``x$ pasa más de 5 horas diarias en clase'', $G(x)=``x$ estudia al menos 3 horas fuera de clase diariamente'' y $P(x)=``x$ x asiste a todas las clases sin faltar''.}\\

\textbf{Traduzca al español la siguiente proposición:}
\[
(\exists x \in X, P(x) \land G(x)) \rightarrow (\exists y \in X, H(y) \land (\forall z \in X, G(z) \rightarrow (P(z) \lor \neg H(z)))).
\]

\begin{enumerate}
\item a) Si un primíparo de economía asiste a todas sus clases sin faltar, entonces estudia fuera de clase y pasa más de 5 horas diarias en clase.
\item b) Si algún primíparo de economía pasa más de 5 horas diarias en clase y estudia fuera de clase, entonces todos los primíparos que estudian fuera de clase también asisten a todas sus clases sin faltar.
\item c) Si existe al menos un primíparo de economía que pasa más de 5 horas diarias en clase y estudia fuera de clase, entonces hay un primíparo que asiste a todas sus clases sin faltar y todos los primíparos que estudian fuera de clase pasan más de 5 horas en clase o no asisten a todas sus clases sin faltar.
\item d) Si un primíparo de economía estudia fuera de clase, entonces o bien asiste a todas sus clases sin faltar o bien no pasa más de 5 horas en clase.
\end{enumerate}
}
\end{question}


\end{document}
